\documentclass{article}
\usepackage[utf8]{inputenc}

\title{ZK Bootcamp: Homework 1}
\author{Immanuel Segol}
\date{July 2022}

\begin{document}
\maketitle

\section*{Maths introduction}
\subsection*{Problem 1}
Working with the following set of Integers $S = \{0,1,2,3,4,5,6\}$, what is:
\begin{enumerate}
\item $4+4 \equiv 1 \pmod 7$
\item $3 \cdot 5 \equiv 1 \pmod 7$
\item Using Fermat's Little Theorem we can find the inverse of 3, $3^{-1} \equiv 3^{7-2} \pmod 7 \equiv 5 \pmod 7$
%\item $3^{-1} \equiv 5$ 
\end{enumerate}
\subsection*{Problem 2}
For $S =\{0,1,2,3,4,5,6\}$, can we consider S and operation + to be a group ?
\begin{enumerate}
\item Closure, for any ${a,b} \in S$, we can see that $a+b \pmod 7 \in S$
\item Associativity, since $S \subseteq \mathbb{Z}$ thus $(a+b)+c = a+(b+c)$
\item Identity Element, $0 \in S$ thus for $a \in S,  a + 0 = a$
\item Inverse element.  $-a \equiv 5a \pmod 7$ and $-a + 5a = 4a \pmod 7 \in S$
\end{enumerate}
\subsection*{Problem 3}
What is $-13 \pmod 5$ ?
\begin{list}
    \item $-13 \pmod 5 \equiv -2 \pmod 5 \equiv 2$
\end{list}
\end{document}